\section{Introduzione}\label{sec:introduzione}
\subsection{Pendolo invertito}\label{subsec:intro-pendolo}
  Il pendolo invertito su rotaia è un sistema classico studiato dalla teoria del controllo.
  Il problema che ci si pone è il seguente:
  \begin{framed}
  \emph{
      Un pendolo è posto su un carrello libero di muoversi orizzontalmente su una rotaia.
      Il carrello è dotato di un motore che gli permette di accelerare. Conoscendo lo stato del sistema
      $\bf x$, trovare un espressione per la forzante esercitata dal motore $\bf u = \bf u(\bf x)$ che faccia sì
      che il pendolo rimanga orientato verso l'alto.
    }
  \end{framed}

  Una descrizione formale del sistema è data dal diagramma in fig. %todo add ref
  % todo add figure
  dove $\mathbf y$ è l'output del sistema, $\mathbf x$ e $\bf f$ sono rispettivamente
  lo \emph{stato interno} e la \emph{legge di evoluzione} del sistema. $\mathbf y$, $\mathbf x$ e $\bf f$ sono
  \emph{vettori colonna}.

\subsection{Controller \textsc{LQG}}\label{subsec:intro-lqg}
Per controllare un sistema di questo tipo (\ref{subsec:intro-pendolo}), dobbiamo tenere presente dei seguenti vincoli:
\begin{enumerate}
  \item%
    La forzante $\bf u$ ha un certo costo associato.
    Questo è da intendersi sia come costo \emph{materiale} (i.e. il costo della corrente per mettere in moto
    il motore) ma, soprattutto, anche come costo \emph{fisico} (i.e. non si può realizzare un motore che ha potenza
    infinita).
  \item%
    Si può osservare solamente una parte $\bf y$ dello stato $x$ del sistema.
    Le osservazioni, inoltre, sono soggette a un certo rumore $\bf w_d$.
\end{enumerate}

Un controller \textsc{LQG}, definito in (REFF) permette di risolvere questi vincoli in modo ottimale, grazie
alla proposizione (reff). %todo qui ci va una ref
% also todo, magari cita il libro di Brunton

\begin{framed} %todo fai qualcosa di carino
  Un controller \textsc{LQG} è un sistema dinamico con input $\bf y$, output $\bf u$ e stato interno
  $\hat{\bf x}$:
  \begin{equation}
    \begin{aligned}[c]
      \frac d {dt} \hat {\bf x} &= (A-K_f C - B K_r) \hat{\bf x} + K_f \bf y \\ %fixme matrices need to be not itlaics
      \bf u               &= -K_r \hat{\bf x}
    \end{aligned}
    \label{eq:lqg}
  \end{equation}
\end{framed}
\begin{framed} %todo fai qualcosa di carino
  Il controller descritto in %todo ref
  minimizza la seguente funzione di costo, mediata su molte realizzazioni del rumore $\bf w_d$:
  \begin{equation}
    J(t) \doteq \left<
      \int_0^t  ds\ \bf{x} (s)^* Q \bf {x} (s) + \bf {u} (s)^* R \bf {u} (s) %todo t o +\infty?
      % fixme capire come mai bf sminchia tutto
    \right>
    \label{eq:lqg-costo}
  \end{equation}
\end{framed}

Un controller \textsc{LQG} equivale all'unione di un controller \textsc{LQR} e di un filtro di Kalman.
Per approfondimenti, consultare %todo bib ref


\subsection{Controller \textsc{PID}}\label{subsec:intro-pid}
%todo da fare nel caso in cui questa cosa mi serva davvero
