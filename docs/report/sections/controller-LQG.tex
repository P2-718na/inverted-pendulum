\section{Sviluppo del controller \textsc{LQG}}\label{sec:lqg}
\subsection{Controllabilità del sistema}\label{subsec:controllabilità}
Prima di procedere con la creazione del controller, dobbiamo accertarci che il sistema sia \emph{controllabile}.
Condizione sufficiente per la controllabilità è che la \emph{matrice di controllabilità}, definita in \eqref{eq:matrice-controllabilità}
abbia rango massimo.

\begin{framed}
  \textbf{DEF} Matrice di controllabilità
  \begin{equation}
    \mathcal C \doteq \left[
    \begin{matrix}
    B & AB & A^2B & \ldots & A^{n-1}B
    \end{matrix}
    \right]
    \label{eq:matrice-controllabilità}
  \end{equation}
\end{framed}

Svolgendo i calcoli, si trova che il sistema \eqref{eq:A-e-B} è sempre controllabile\footnote{Assumendo che tutti i
parametri siano positivi.}.

\subsection{Discretizzazione del sistema}\label{subsec:discretizzazione}
Nella sezione \ref{subsec:intro-lqr} ho spiegato come funziona un controller \textsc{lqr} per sistemi a \emph{tempo continuo}.
Il sistema che ho realizzato, tuttavia, non può raccogliere dati dai sensori in modo continuo: tra un osservazione
$\mathbf y_k$ e la successiva $\mathbf y_{k+1}$ passa un certo intervallo di tempo $\Delta t$.
Tutte le considerazioni fatte finora continuano a valere, ma dal punto di vista pratico è necessario discretizzare il sistema.
Vogliamo trovare le matrici $A_d$ e $B_d$ tali cui le equazioni
\begin{equation}
  \begin{aligned}
    \mathbf{x}_{k+1} = A_d \mathbf{x}_k + B_d f_k
  \end{aligned}
  \label{eq:moto-discreto}
\end{equation}
approssimino meglio le \eqref{eq:state-space} fissato un $\Delta t > 0$.
Si può dimostrare\cite{brunton2019data}
che valgono:
\begin{equation}
  \begin{aligned}
    A_d &= e^{A\Delta t} \\
    B_d &= B\left( \int_0^{\Delta t} e^{A \tau} d\tau \right)
  \end{aligned}
  \label{eq:discrete-mapping}
\end{equation}
Tuttavia, operativamente, per $\Delta t \ll 1$, possiamo trovare velocemente una soluzione approssimata per \eqref{eq:discrete-mapping}:
\begin{equation}
  \begin{aligned}
    \mathbf x_{k+1} &\approx x_k + \dot x|_k \Delta t\\
      &\approx \mathbf x_k + (A \mathbf x_k + B f_k) \Delta t\\
      &\approx (I + \Delta t A) \mathbf x_k + B \Delta t f_k\\
  \end{aligned}
  \label{eq:discrete-mapping-approx}
\end{equation}
Le matrici che cerchiamo sono quindi:
\begin{equation}
  \begin{aligned}
  A &=
  \left(\begin{array}{cccc}1&\Delta t&0&0\\0&1&-\frac{3gm}{m+4M}\Delta t&0\\0&0&1&\Delta t\\0&0&-\frac{6g(m+M)}{l(m+4M)} \Delta t&1\\\end{array}\right)
  \\
  B &=
  \left(\begin{array}{c}0\\\frac{4}{m+4M} \Delta t\\0\\\frac{6}{lm+4lM} \Delta t\\\end{array}\right)
  \label{eq:Ad-e-Bd}
  \end{aligned}
\end{equation}

\subsection{Calcolo dei coefficienti del controller}\label{calcolo-coefficienti}
Definisco la funzione costo per il sistema \eqref{eq:discrete-mapping}:
\begin{equation}
  J(t) =
  \sum_{k=0}^{+\infty} \left[ \mathbf{x}_k^* Q \mathbf {x}_k + f_k R f_k \right]
  \label{eq:lqr-costo-discreto}
\end{equation}

%todo spiega come mai c'è da risolvere equazione di ricciati
