\section{Modello e implementazione del sistema}\label{sec:modello}
\subsection{Modello del sistema}\label{subsec:modello-del-sistema}
Per studiare il sistema, devo ricavare le equazioni del moto e devo linearizzarle attorno ai punti di equilibrio.
Terrò conto delle seguenti assunzioni:

\begin{enumerate}
  \item%
  Voglio risolvere il problema del pendolo invertito solo per condizioni iniziali prossime ai punti di equilibrio
  (strategie di \emph{swing-up} e \emph{swing-down} non sono oggetto del mio studio.
  \item%
  Gli attriti sono trascurabili.
  \item%
  Il sistema è controllabile (\emph{Dimostrerò} questa assunzione nella sezione %todo add ref)
\end{enumerate}

Il sistema è descritto in figura %todo add figura.
Notare che la variabile generalizzata $x$ e lo stato del sistema $\bf x$ sono due cose diverse. Non ci sarà confusione,
visto che una è scritta in grassetto e l'altra no.

La Lagrangiana del sistema non lineare è:
\begin{equation}
  \mathcal L = \frac 1 2
  \label{eq:lagrangiana}
\end{equation}

Risolvendo le equazioni di Eulero trovo le equazioni del moto:
\begin{equation}
  \left\{
    \begin{aligned}
      \ddot x &= culo \\
      \ddot \theta &= culo
    \end{aligned}
  \right.
  \label{eq:eq-moto}
\end{equation}

Riduco di uno l'ordine del sistema.
\begin{equation}
  \dot {\bf x} = \bf f =
  \left\{
    \begin{aligned}
      \dot x &= v \\\textsc{lqr}
      \dot v &= culo \\
      \dot \theta &= \omega \\
      \dot \omega &= culo
    \end{aligned}
  \right.
  \label{eq:eq-1-grado}
\end{equation}

I punti di equilibrio del sistema \eqref{eq:eq-1-grado} sono:
\begin{equation}
  \begin{aligned}
    \bf x_I &= (x, 0, 0, 0) \to & \text{instabile} \\
    \bf x_S &= (x, 0, \pi, 0) \to & \text{stabile}
  \end{aligned}
  \label{eq:punti-equilibrio}
\end{equation}

Linearizzo le equazioni \eqref{eq:eq-1-grado} attorno a $\bf x_I$.
Calcolo la Jacobiana:
\begin{equation}
  J_{\bf f}(\bf x_I) =
  \left(
    \begin{matrix}
      0 & 1 & 1 & 1 & 1\\
      1 & 0 & 1 & 1 & 1\\
      1 & 1 & 0 & 1 & 1\\
      1 & 1 & 1 & 0 & 1
    \end{matrix}
  \right)
  \label{eq:jacobiana}
\end{equation}

Il sistema linearizzato può essere scritto nella forma $\dot {\bf x} = A\bf x + B \bf u$, dove $A$ e $B$ valgono:
\begin{equation}
  A =
  \left(
  \begin{matrix}
    0 & 1 & 1 & 1\\
    1 & 0 & 1 & 1\\
    1 & 1 & 0 & 1\\
    1 & 1 & 1 & 0
  \end{matrix}
  \right)
  \hspace{10px}
  B =
  \left(
  \begin{matrix}
    1\\
    1\\
    1\\
    0
  \end{matrix}
  \right)
  \label{eq:A-e-B}
\end{equation}

\subsection{Implementazione del sistema}\label{subsec:implementazione-del-sistema}
Ho realizzato il sistema modellato in sezione \ref{subsec:modello-del-sistema}, così come descritto in fig. %todo add figura
La scelta dei componenti garantisce che l'attrito sia trascurabile.
Una serie di sensori permette di misurare lo stato del sistema.
Per maggiori dettagli sui componenti utilizzati, consultare l'appendice %todo add ref
